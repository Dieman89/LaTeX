\section{Future Improvements}
\label{s:Future-improvements}
In this section, there will be discussions about future improvements that could be
done in a future where time was not a constraint.

\subsection{Headless CMS}
\label{s:Headless-CMS}
In order to separate the content from the frontend and backend, an Headless CMS
would result in a very ideal choice. This would make the content management much
more easier to maintain and to work with in a large scale as the content would be
fetched through a third party API rather then having it in the source code.

\subsection{CI/CD}
\label{s:CI-CD}
It would be very nice to not have to manually run things locally and have an
automated pipeline to run tests in the continuous integration environment and
deploy a new version of the service to a non-prod and prod environment. This
whole process not only create the base of a service that would run on a server
continously but also reduce manual errors from developers that would then only
focus on writing code rather than monitoring the behaviour of the service in
production.

\subsection{Terraform}
\label{s:Terraform}
Terraform is a tool used to manage an infrastructure on AWS, Azure or Google
Cloud Platform. It is known as infrastracture as code as it is used to defined
the infra in configuration files, run the plan that would review and make
changes to the infrastructure through the public API and finally apply the
provision and update the state files on the remote servers.

\subsection{Docker}
\label{s:Docker}
The benefits of containerisation are that it is easier to manage and it is
easier to vertically scale the application. A containerised application uses less
memory, start much quickly and enabled portability.

\subsection{Software as a Service}
\label{s:SaaS}
It could be a nice idea to have the tool running on a distributed system that
could operate on the web on demand. This would enable developers in need of fast
documentation to have quick access and a portal up and running in a third party
application with just a schema as a requirement. An integration would then be
easily supported through GitHub as a registry to update and deliver schema
evolutions automatically.

\subsection{Integrated Frontend}
\label{s:Integrated-Frontend}
This would be part of the previous section, as once the schema is uploaded, it
would require a frontend framework that render the files and display it to the
end users.

\subsection{Pure Functional Programming}
\label{s:Pure-Functional-Programming}
Probably a refactoring in another language that is more scalable and safe such
as Scala with libraries such as Cats and Cats Effect would be a better choice
and a safe bet for a tool used in production.

\section{Tool Used}
\label{s:Tool-Used}
This section will expand on the tool used to create the project and the report.
The text editor being used is \texttt{Visual Studio Code}.

To avoid building everything manually on each change I am using a collection of
tools that are well-integrated with my text editor: LaTeX Workshop, LaTeX
Utilities, Zotero, LaTeX Better BibTeX

LaTeX Workshop is probably the most important of all. It automatically builds
and watches the output pdf file to have a nice hot-reload without closing and
re-opening the file to see the changes in a very short delay time.

LaTeX Utilities is an addon for LaTeX Workshop that expands some
functionalities such as Zotero citation integrations and smart paste formatter
for the images.

Zotero does not only come in a form of a plugin, but also as Software for
references and citations. This plugin will attach to the Zotero server and pull
data from their endpoint to easily access the library.

Better BibTeX automates the export of the library and keeps it automatically
updated on each change (addition, removal, or change). This will export a nice
\texttt{.bib} file that can then be used in LaTeX.


\section{Reflection}
\label{s:Reflection}
This section will contain my reflection

