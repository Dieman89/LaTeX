Generating an up-to-date documentation from a specific source of truth such as a
schema, is a very important aspect of documentating an Application Programming
Interface (API) for a large corporate company and not only. The absence of a
tool to generate such documentation, more specifically a tool to generate
Markdown files. Even more importantly, one that could generate a file with a
support for syntax aimed at custom components such as Markdown for the component
era (MDX). The modern world is building the majority of their web in React, and
it is very important and relevant that a file supports both Markdown (MD) and
JavaScript XML (JSX) and MDX does just that. This would certainly mean and
resolve one of the biggest issues developers are finding when working with
GraphQL APIs, which is Introspection. Introspection is a GraphQL query that
allows developers to navigate and discover the traits of an entire Schema from
the external world, which is something we all want to avoid at all cost for a
security perspective. This project has been designed and researched and
discussed with my supervisor to be the solution at the above problems while
keeping it framework agnostic.

\section{Aims}
\label{s:Aim}

This project will not only generate the markdown files but also give examples on
how to integrate them in a frontend framework of choice.

\section{Objectives}
\label{s:Objectives}

\begin{description}

  % \item[Chapter~\ref{c:Introduction}] explained explain the project at a high level language.

  \item[Chapter~\ref{c:Ethical-Considerations}] will explain the legal, social, and ethical aspects of the project.

  \item[Chapter~\ref{c:Literature-Review}] will give an overview of the literature review.

  \item[Chapter~\ref{c:UX-Design-Agile-Methodologies}] will give an overview of the user experience and agile methodologies.

  \item[Chapter~\ref{c:Implementation}] will give an overview of the implementation.

  \item[Chapter~\ref{c:Testing}] will give an overview of the testings.

  \item[Chapter~\ref{c:Conclusions}] will give an overview of the conclusions.

\end{description}
