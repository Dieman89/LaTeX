\section{Synopsis}
\label{s:Synopsis-5}
In modern environment, delivering innovative idea in a fast and reliable manner
is extremely significant for any organizations. In the existing scenario,
Insurance industry need to better respond to dynamic market requirements, faster
time to market for new initiatives and services, and support innovative ways of
customer interaction. In past few years, the transition to cloud platforms has
given benefits such as agility, scalability, and lower capital costs but the
application lifecycle management practices are slow with this disruptive change.
DevOps culture extends the agile methodology to rapidly create applications and
deliver them across environment in automated manner to improve performance and
quality assurance. Continuous Integration (CI) and Continuous delivery (CD) has
emerged as a boon for traditional application development and release management
practices to provide the capability to release quality artifacts continuously to
customers with continuously integrated feedback. The objective of the paper is
to create a proof of concept for designing an effective framework for continuous
integration, continuous testing, and continuous delivery to automate the source
code compilation, code analysis, test execution, packaging, infrastructure
provisioning, deployment, and notifications using build pipeline concept.

\section{Useful Quotes}
\label{s:Useful-Quotes-5}
Application development phases include coding, building, integrating, testing,
troubleshooting, infrastructure provisioning, configuration management, setting
up runtime environment, and deploying applications in different environments \citep{soniEndEndAutomation2015}.

Continuous Delivery is a software development discipline where you build a
deployment pipeline in such a way that the software can be released to
production at any time \citep{soniEndEndAutomation2015}.

Quality gates helps to manage orchestration across build pipeline. Ideally, if
one step fails to execute successfully then continuous delivery or next step
must not take place \citep{ soniEndEndAutomation2015}.

Every commit into source code repository by a developer is verified by automated
build process using continuous integration server \citep{soniEndEndAutomation2015}.

Rapid delivery in Dev, Test, Pre-production and Production environments:
Elimination of manual and repeatable processes based deployment \citep{soniEndEndAutomation2015}.

\section{Personal Reflection}
\label{s:Personal-Reflection-5}
This is an important step for my project. Since one of the requirements is about
evaluating if the final product is successfull or not it will require end to end
testings and a good pipeline in a ci/cd workflow.

\section{Question Raised}
\label{s:Question-Raised-5}
How many envoirments do I want in my pipeline? How many steps?
Should I consider pre-hooks with husky for linting and more? E2E and Unit?
Should I use Vercel for CD?

\section{Notes}
\label{s:Notes-5}
Need to come up with a good architecture and a plan to avoid messy pipelines.
