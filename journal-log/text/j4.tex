\section{Synopsis}
\label{s:Synopsis-4}
This paper provides a systematic comparison between two well-known Agile
methodologies: Scrum, which is a framework of doing projects by allocating tasks
into small stages called sprints, and Kanban, which is a scheduling system to
manage the flow of work by means of visual signals. In this regard, both
methodologies were reviewed to explore similarities and differences between them.
Then, a focus group survey was performed to specify the preferable methodology
for product development according to various parameters in the project
environment including project complexity, level of uncertainty, and work size
with consideration of output factors like quality, productivity, and delivery.
Results show the flexibility of both methodologies in approaching Agile
objectives, where Scrum emphasizes on the corporation of the customer and
development teams with a focus on particular skills such as planning,
organization, presentation, and reviewing which makes it ideal for new and
complex projects where a regular involvement of the customer is required,
whereas Kanban is more operative in continuous-flow environments with a steady
approach toward a system improvement.

\section{Useful Quotes}
\label{s:Useful-Quotes-4}
Scrum is a project management approach that works according to iterative and
incremental developments \citep{zayatFrameworkStudyAgile2020}.

The stages of Scrum model start with product backlog creation as the first stage
\citep{zayatFrameworkStudyAgile2020}.

Nowadays, Scrum and Kanban methodologies are the most adopted methods by system
devel opment organizations around the world \citep{zayatFrameworkStudyAgile2020}.

Scrum is defined as a framework in which people can deal with complex problems,
while maintaining high productivity in delivering high-quality products
\citep{zayatFrameworkStudyAgile2020}.

Kanban, which means visual signal, was firstrst used by workers in Toyota to track
processes on their manufacturing system \citep{zayatFrameworkStudyAgile2020}.

The main focus of Kanban is the flow of work along with the elimination
of unnecessary activities which leads to shorter feedback loops \citep{
zayatFrameworkStudyAgile2020}.

\section{Personal Reflection}
\label{s:Personal-Reflection-4}
While working on my project, I will need a good agile framework to split the work in epics
and evaluate what needs to be researched through spikes and how many stories are needed to complete
each epic. Even though Scrum could be a nice addition to the workflow, it is most likely aimed to a group of
at least two. I will have to decide if I want to simulate being in a small group or use Kanban instead avoiding
unnecessary cerimonies that Scrum would lock me in.

\section{Question Raised}
\label{s:Question-Raised-4}
How would I do standups?
Would I need to do retros fortnightly?
Who would refine my tickets and point my stories?

\section{Notes}
\label{s:Notes-4}
I have to think about which Agile framework to use. The questions that are previously made will
probably lead me to use Kanban to work more efficiently, but this paper is nice to evaluate and choose
the methodology that will follow me during this project.
