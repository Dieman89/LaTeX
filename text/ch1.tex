This is where I will write the introduction

\section{Generate Documentation}
\label{s:Generate-Documentation}

Explain more about this project specifially at the use of the tool.

\section{Aims}
\label{s:Aim}

This project will not only generate the markdown files, but also give an examples on how to integrate them in a frontend framework of choice, in this case NextJS.

\section{Objectives}
\label{s:Objectives}

\begin{description}

  \item[Chapter \ref{c:Introduction}] will explain the project in a general way.

  \item[Chapter \ref{c:Ethical-Consideration}] will explain the legal, social and ethical aspects of the project.

  \item[Chapter \ref{c:Literature-Review}] will give an overview of the literature review.

  \item[Chapter \ref{c:UX-Design-Agile-Methodologies}] will give an overview of the user experience, design and agile methodologies.

  \item[Chapter \ref{c:Implementation}] will give an overview of the implementation.

  \item[Chapter \ref{c:Testing}] will give an overview of the testings.

  \item[Chapter \ref{c:Conclusions}] will give an overview of the conclusions.

\end{description}



