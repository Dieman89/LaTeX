\section{Synopsis}
\label{s:Synopsis-3}
New Internet-enabled devices and Web services are introduced on a daily basis.
Documentation formats are available that describe their functionalities in terms
of API endpoints and parameters. In particular, the OpenAPI specification has
gained considerable influence over the last years. Web-based solutions exist
that generate interactive OpenAPI documentation with HTML5 \& JavaScript. They
allow developers to quickly get an understanding what the services and devices
do and how they work. However, the generated user interfaces are far from
real-world practices of designers and end users. We present an approach to
overcome this gap, by using a model-driven methodology resulting in
state-of-the-art responsive Web user interfaces. To this end, we use the
Interaction Flow Modeling Language (IFML) as intermediary model specification to
bring together APIs and frontends. Our implementation is based on open standards
like Web Components and SVG\@. A screencast of our tool is available at https:
//youtu.be/KFOPmPShak4

\section{Useful Quotes}
\label{s:Useful-Quotes-3}
Swagger UI is an open source software that automatically generates a
HTML5-based visualization and interaction frontend from an OpenAPI file \citep{
korenExploitationOpenAPIDocumentation2018}.

Web Components are a recent manifestation of the increased adoption of
component-based software engineering in the frontend Web. Similar approaches on
a JavaScript framework level are Ember, React and Vue.js \citep{
korenExploitationOpenAPIDocumentation2018}.

We strongly believe that automation is one of the keys to tackle challenges in
the creation of situational applications for the long tail \citep{
korenExploitationOpenAPIDocumentation2018}.

\section{Personal Reflection}
\label{s:Personal-Reflection-3}
This paper will be very useful to have a metric of comparison between OpenAPI and Swagger and the
fact that GraphQL doesn't support it. Especially, as previously stated, when introspection is disabled
then nothing would be available to document the schema.
Another point is that even if GraphQL would support Swagger UI, it would still be locked behind proprietary code
without any flexibility or power over the presentation and what to show or hide.
This will be one of the strong point of my project, since the files generated will be markdown
that can be parsed and manipulated as per developer liking.

\section{Question Raised}
\label{s:Question-Raised-3}

\section{Notes}
\label{s:Notes-3}
Make this one of the strong points.
