\section*{Introduction}
\label{s:Introduction}
API documentation is a very important part of the software development process,
it improves the dev experience and makes it easier to integrate, improves
maintability and easily enables versioning with indication on deprecated fields
\citep{fanWhyAPIDocumentation2021}. Working with API documentation is not always
straightforward, especially when the working with gateway and federations, as it
would require much effort from all the developers involved in the process.
Swagger makes documentation for REST-APIs very straightforward, as it
automatically generates HTML based visualisation and interaction out of the box
for any consumer of the API \citep{korenExploitationOpenAPIDocumentation2018}.
Since Facebook released GraphQL to the mass in 2015, many individual developers
and companies are switching and converting to it to build their own APIs as it
enables them to have a much more flexible and efficient way of building their
APIs \citep{britoRESTVsGraphQL2020}. GraphQL gives to the consumer only the data
that he needs, solving the overfetching and underfetching problem related to the
traditional REST-APIs \citep{witternGeneratingGraphQLWrappersREST2018}. Querying
the API the way that user want, enables for so much flexibility but can also be
a threat to security if not handled properly. The objective of this project is
to enable API producers to build a secure GraphQL APIs without recheable
introspection on endpoint while still count on a tool that can generate a
framework agnostic structured documentation.

\section*{Problem Domain}
\label{s:Problem-Domain}
When working on such a problem, there are many different complications as having
updated documentation for GraphQL APIs, one of them - and probably the most
important - being security. Introspection enables users to query a GraphQL API
and discover its schema structure, giving bad actors a chance to discover
potentially malicious operations \citep{khalilWhyYouShould2021} easily and
disrupt the availability of the API, but it is also a requirement for tools such
as \textit{GraphiQL} and \textit{Playground}. This makes this a severe dilemma
for producers who want to keep their APIs as secure as possible, away from
indiscreet eyes, and open to potential threats but still have documentation
tooling.

\section*{Methodology}
\label{s:Methodology}

\section*{Evaluation}
\label{s:Evaluation}
