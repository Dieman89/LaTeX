\section*{Introduction}
\label{s:Introduction}
This project aims at solving the choice between security and programmatically
generated documentation in the world of the GraphQL Application Programming
Interface (API). API documentation is an essential part of the software
development process as it improves the dev experience --- making it easier to
integrate --- enhancing maintainability and conveniently enables versioning with
indication on deprecated fields \citep{fanWhyAPIDocumentation2021}. Working with
API documentation is not always straightforward, especially when working with
gateway and federation, as it would require much effort from all the developers
involved in the process. Swagger makes documentation for Representational state
transfer (REST) APIs very uncomplicated, as it automatically generates HyperText
Markup Language (HTML) based visualisation and interaction out of the box for
any consumer of the API \citep{korenExploitationOpenAPIDocumentation2018}. Since
Facebook released GraphQL to the mass in 2015, many individual developers and
companies are switching and converting to it to build their APIs, as it enables
them to have a much more flexible and efficient way of building their APIs
\citep{britoRESTVsGraphQL2020}. GraphQL gives to the consumer only the data that
he needs, solve the overfetching and underfetching problem related to the
traditional REST APIs \citep{witternGeneratingGraphQLWrappersREST2018}.
\begin{figure}[H]
  \centering
  \includegraphics[width=0.6\textwidth]{figures/restvsgraph}
  \caption{REST API vs GraphQL API}
  \label{f:restvsgraph}
\end{figure}
Querying the API the way users want enables flexibility but can also threaten
security if not appropriately handled. This project aims to facilitate API
producers to build secure GraphQL APIs without reachable introspection on the
endpoint while still counting on a tool that can generate framework-agnostic
structured documentation.

\section*{Problem Domain}
\label{s:Problem-Domain}
When working on such a problem, there are many complications in keeping
constantly automated and updated documentation for GraphQL APIs. One of them is
security. Introspection enables users to query a GraphQL API and discover its
schema structure, giving bad actors a chance to find potentially malicious
operations \citep{khalilWhyYouShould2021} quickly and disrupt the availability
of the API. However, it is also a requirement for tools such as
\textit{GraphiQL} and \textit{Playground}. This is a severe dilemma for
producers who want to keep their APIs as secure as possible, away from
indiscreet eyes, and closed to potential threats but still, have documentation
tooling. If the attackers have access to the whole schema through introspection,
it will be effortless to find and exploit API calls meant for internal use and
debugging purposes \citep{rizwanGraphQLCommonVulnerabilities2021}. Through the
same technique, the attackers could also get access to mutations and API calls
intended to add, edit or delete specific data on the database, making it a real
threat. Many other security issues are linked to the activation of the
introspection and misconfiguration; some are information disclosure, insecure
direct object references and inexistent Access Control List (ACL) \citep{
yeswehackHowExploitGraphQL2021}. By design, GraphQL has a fetching inefficiency
known as \textit{N+1 Problem} where the number of queries executed against the
database (or other upstream services) can be as large as the number of nodes in
the resulting graph \citep{ graphqlbypopSuppressingProblemGraphQL2020}.
\begin{figure}[H]
  \centering
  \includegraphics[width=0.5\textwidth]{figures/code/n+1}
  \caption{GraphQL N+1 Problem}
  \label{f:GraphQL-N1-Problem}
\end{figure}
In the example above, the query against the schema would make a single call to
the database to retrieve the first N students, and then for each of these Ns
students it would make a separate query to the same database to fetch M friends
details (N calls), hence N+1. Having introspection disabled is the right choice
looking at a security perspective, and this project will help solve the downside
of not having tools to help document the API and more.

\section*{Methodology}
\label{s:Methodology}
The project will be divided into two main parts, one being the package to build
the structure and generate the documentation from a single source of truth, in
this case is a GraphQL schema, and the other is the implementation of a
statically generated website that renders the previously generated file in HTML
format \citep{gagliardiDjangoRESTMeets2021}. NodeJS will be used as the backend
runtime for the backend to generate the documentation that will then be served
to the frontend. The frontend will use NextJS as a React framework to generate
the static HTML files parsing previously generated markdowns. A JAMstack
(Javascript, API and Markup) with Headless CMS (Content Management System) the
approach will be used throughout this project, with additional content outside
the generation scope, being decoupled from the whole Versioning Control System
(VCS) and being fetched on build time through API queries
\citep{zammettiWhatJAMstackAll2020}.
\begin{figure}[H]
  \centering
  \includegraphics[width=0.5\textwidth]{figures/jamstack}
  \caption{JAMstack Workflow}
  \label{f:jamstack}
\end{figure}
A Kanban board will be used to maximise efficiency and visualise the workload
and limit the WIP of the project. Kanban has been chosen over Scrum as it is
more flexible and does not require running ceremonies, including daily standups.
Also, Scrum is overkill for a team of one developer as it would take much time
lost on stories, refinements, pointing, spikes and epics while also managing
them over different sprints \citep{zayatFrameworkStudyAgile2020}. On the other
hand, Kanban will focus much more on delivery and is more suited for a small
team or individual developers.

The programming language will be JavaScript for the Proof of Concept (PoC) that
will then be refactored in TypeScript to make the development process more
efficient with its powerful type system saving developers time
\citep{freemanUnderstandingTypeScript2021}. The architecture will be managed
with popular tools such as GitHub for versioning control and a single source of
truth for input files, CircleCI for continuous integration, Vercel for
deployment and hosting. To increase package flexibility, the project will also
be containerised building an image through Docker that can run on Amazon Web
Services (AWS) such as Amazon Elastic Compute (ECS) with a virtual remote
machine (EC2) \citep{pratapyadavFormalApproachDocker2021}.

\section*{Evaluation}
\label{s:Evaluation}
A goal-setting methodology will be used to evaluate the project; the initial
goal and key results will be set up at the start of the project with fortnightly
updates. To see if the progress towards the goal is not on the decline, key
performance indicators (KPIs) will be set to monitor any gaps and focus the
attention on the previously set Objectives and Key Results (OKRs). This will
help adjust the goals based on the progress and be able to evaluate if the
project reaches a complete state at the end of the deadlines
\citep{helmoldLeanManagementKPI2020}. On the technical side, testing will ensure
that the end product is performant, accessible and meets the expectations set at
the start of the development process. The project will be tested through a
mixture of Unit and End to End (E2E) tests integrated into the previously stated
CircleCI pipeline before shipping and deploying the code in production
environments. This will ensure that bugged code is only present in
non-production environments, and the last deploy step is reached only when the
whole project is ready for production \citep{yuUtilisingCIEnvironment2020}. The
development process will be evaluated as successful if the OKRs reaches green
metrics and the final project complies with the specifications that have been
set \citep{helmoldLeanManagementKPI2020}. The final product will generate a
structured folder with markdown files that document the references of a GraphQL
schema and are rendered with a JAMstack philosophy.
\begin{figure}[H]
  \centering
  \includegraphics[width=0.7\textwidth]{figures/architecture}
  \caption{High Level Architecture}
  \label{f:architecture}
\end{figure}
