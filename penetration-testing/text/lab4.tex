
\section{Task 1}
\label{lab4-task1}

Below a table with the CWE top 25 weaknesses in the 2021.
\begin{table}[H]
  \begin{center}
    \label{tab:lab4-table1}
    \begin{tabular}{c c c}
      \toprule
      \textbf{Number} & \textbf{ID} & \textbf{Description} \\
      \midrule
      1 & CWE-787 & Out-of-bounds Write\\
      2 & CWE-79 & Cross-site Scripting\\
      3 & CWE-125 & Out-of-bounds Read\\
      4 & CWE-20 & Improper Input Validation\\
      5 & CWE-78 & OS Command Injection\\
      6 & CWE-89 & SQL Injection\\
      7 & CWE-416 & Use After Free\\
      8 & CWE-22 & Path Traversal\\
      9 & CWE-352 & Cross-Site Request Forgery\\
      10 & CWE-434 & Unrestricted Upload of File\\
      11 & CWE-306 & Missing Authentication\\
      12 & CWE-190 & Integer Overflow\\
      13 & CWE-502 & Deserialization of Untrusted Data\\
      14 & CWE-287 & Improper Authentication\\
      15 & CWE-476 & NULL Pointer Dereference\\
      16 & CWE-798 & Use of Hard-coded Credentials\\
      17 & CWE-119 & Memory Buffer\\
      18 & CWE-862 & Missing Authorization\\
      19 & CWE-276 & Incorrect Default Permissions\\
      20 & CWE-200 & Exposure of Sensitive Information\\
      21 & CWE-904 & Insufficient Logging and Monitoring\\
      22 & CWE-908 & Insufficient Session Management\\
      23 & CWE-912 & Insufficient Encryption\\
      24 & CWE-913 & Insufficient Cryptographic Strength\\
      25 & CWE-914 & Insufficient Cryptographic Protection\\
      \bottomrule
    \end{tabular}
  \end{center}
\end{table}

\subsection{CWE-787: Out-of-bounds Write}
\label{sec:lab4-cwe787}
The table below shows the relationship between the subject CWE-787 and its
relevant weaknesses.

\begin{table}[H]
  \begin{center}
    \label{tab:lab4-table2}
    \begin{tabular}{c c c c}
      \toprule
      \textbf{Nature} & \textbf{Type} & \textbf{ID} & \textbf{Name} \\
      \midrule
      ChildOf & Class & CWE-119 & Improper Restriction of Operations\\
      ParentOf & Variant & CWE-121 & Stack-base Buffer Overflow\\
      ParentOf & Variant & CWE-122 & Heap-base Buffer Overflow\\
      ParentOf & Base & CWE-123 & Write-what-where Condition\\
      ParentOf & Base & CWE-124 & Buffer Underflow\\
      CanFollow & Base & CWE-822 & Untrusted Pointer Dereference\\
      CanFollow & Base & CWE-823 & Use of Out-of-range Pointer\\
      CanFollow & Base & CWE-824 & Access of Uninitialized Pointer\\
      CanFollow & Base & CWE-825 & Expired Pointer Dereference\\
      MemberOf & Category & CWE-1218 & Memory Buffer Errors\\
      \bottomrule
    \end{tabular}
  \end{center}
\end{table}

The integrity and availability are the most common consequences of the
exploitation as the technical impact might be modified memory, DoS and execution
of unauthorised code and commands. Below is a list with related CVE's and their
description.

\begin{description}[align=left]
  \item [CVE-2020-0022] chain: mobile phone Bluetooth implementation does not
  include offset when calculating packet length
  \item [CVE-2009-1010] chain: compiler optimisation removes or modifies code
  used to detect integer overflow
  \item [CVE-2009-0269] chain: -1 value from a function call was intended to
  indicate an error, but is used as an array index instead.
  \item [CVE-2002-4268] chain: integer signedness error passes signed
  comparison, leading to heap overflow.
\end{description}

Some of the potential mitigation can be separated into different sections.

\begin{description}[align=left]
  \item [Requirements] picking a language that does not produce this
  vulnerability is the easiest mitigation that can be applied, for example
  picking a language that have their own memory management such as Scala or C++
  means that are not subject to buffer overflows.
  \item [Build and Compilation] execute the source code using a protection
  mechanism that eliminates buffer overflows.
  \item [Implementation] adhere to a specific set of strict rules when
  allocating or managing the memory.
  \item [Operation] execute the source code using a feature that randomly
  arrange the position of the software executable and libraries in the memory.
\end{description}

Additionally, to detect these weaknesses, an automated static or dynamic
analysis tools could be placed.

\subsection{CWE-79: Cross-site Scripting}
\label{sec:lab4-cwe79}
The table below shows the relationship between the subject CWE-79 and its
relevant weaknesses.

\begin{table}[H]
  \begin{center}
    \label{tab:lab4-table3}
    \begin{tabular}{c c c c}
      \toprule
      \textbf{Nature} & \textbf{Type} & \textbf{ID} & \textbf{Name} \\
      \midrule
      ChildOf & Class & CWE-4 & Injection\\
      ParentOf & Variant & CWE-80 & Basic XSS\\
      ParentOf & Variant & CWE-81 & Neutralization of Script in an Error\\
      ParentOf & Variant & CWE-83 & Neutralization of Script in Attributes\\
      ParentOf & Variant & CWE-84 & Neutralization of Encoded URI\\
      ParentOf & Variant & CWE-85 & Doubled Character XSS Manipulations\\
      ParentOf & Variant & CWE-86 & Neutralization of Invalid Characters\\
      ParentOf & Variant & CWE-87 & Neutralization of Alternate XSS\\
      ParentOf & Chain & CWE-692 & Denylist to Cross-Site Scripting\\
      PeerOf & Composite & CWE-494 & Code Without Integrity Check\\
      CanFollow & Variant & CWE-113 & Neutralization of CRLF Sequences\\
      CanFollow & Base & CWE-184 & List of Disallowed Inputs\\
      CanPrecede & Base & CWE-494 & Code Without Integrity Check\\
      MemberOf & Category & CWE-137 & Data Neutralization Issues\\
      \bottomrule
    \end{tabular}
  \end{center}
\end{table}

The CIA triad is all affected as the attacker will craft a client-side script
that is then parsed by a browser that performs some activities. The script will
then be loaded and run by every user on the website, and since the hand has
access to cookies, also the attacker does. The XSS script can also run arbitrary
code if combined with other vulnerabilities. Additionally, the attacker could
also use obfuscation techniques to hide the script. Below is a list with related
CVE's and their description.

\begin{description}[align=left]
  \item [CVE-2014-8958] the Admin GUI allows XSS through cookies.
  \item [CVE-2017-9764] allows XSS through HTTP header.
  \item [CVE-2014-5198] allows XSS through HTTP Referer header.
  \item [CVE-2008-5770] reflected XSS using the PATH info in an URL.
  \item [CVE-2008-4730] reflected XSS not properly handled.
  \item [CVE-2008-5734] reflected XSS sent through email message.
  \item [CVE-2008-0971] stored XSS in a security product.
  \item [CVE-2008-5249] stored XSS using a wiki page.
  \item [CVE-2006-3568] stored XSS in a guestbook application.
\end{description}

Some of the potential mitigation can be separated into different sections.

\begin{description}[align=left]
  \item [Architecture] use a frontend library that does not allow XSS.
  \item [Implementation] use the appropriate encoding on all non-alphanumeric
  characters.
  \item [Implementation] adhere to a specific set of strict rules when
  allocating or managing the memory.
  \item [Operation] execute the source code using a feature that randomly
  arrange the position of the software executable and libraries in the memory.
\end{description}

Additionally, to detect these weaknesses, an automated static and the use of a
Black Box tests on a CI step. To automatically find a fix vulnerabilities on a
CI step, Snyk.io would be very valuable and efficient when building web services
on both frontend and backend, and can be placed used as an action or orb in the
continuous-integration pipeline.

\subsection{CWE-125: Out-of-bounds Read}
\label{sec:lab4-cwe125}
The table below shows the relationship between the subject CWE-125 and its
relevant weaknesses.

\begin{table}[H]
  \begin{center}
    \label{tab:lab4-table4}
    \begin{tabular}{c c c c}
      \toprule
      \textbf{Nature} & \textbf{Type} & \textbf{ID} & \textbf{Name} \\
      \midrule
      ChildOf & Class & CWE-119 & Improper Restriction of Operations\\
      ParentOf & Variant & CWE-126 & Buffer Over-read\\
      ParentOf & Variant & CWE-127 & Buffer Under-read\\
      CanFollow & Base & CWE-822 & Untrusted Pointer Dereference\\
      CanFollow & Base & CWE-823 & Use of Out-of-range Pointer Offset\\
      CanFollow & Base & CWE-824 & Access of Uninitialized Pointer\\
      CanFollow & Base & CWE-825 & Expired Pointer Dereference\\
      MemberOf & Category & CWE-1218 & Memory Buffer Errors\\
      \bottomrule
    \end{tabular}
  \end{center}
\end{table}

In this instance, confidentiality is the only affected aspect, and it is
compromised by reading out-of-bounds memory that would give access to personal
information to the attacker.

\begin{description}[align=left]
  \item [CVE-2014-0160] chain: 'Hearthbleed' bug receives an incosistent
  length paramenter enabling an out-of-bounds read.
  \item [CVE-2009-2523] chain: product does not handle when an input string is
  not NULL terminated.
  \item [CVE-2004-0112] out-of-bounds read due to miproper length check.
  \item [CVE-2004-0183] packet with large number of specified elements cause
  out-of-bounds read.
  \item [CVE-2004-0184] out-of-bounds read, resultant from integer underflow.
  \item [CVE-2004-1940] large length value caused out-of-bounds read.
\end{description}

Some of the potential mitigation can be separated into different sections.

\begin{description}[align=left]
  \item [Implementation] assume all input is malicious, building and
  engineering an acceptable validation strategy.
  \item [Architecture] use a language that provides memory abstractions.
\end{description}

Since the only languages affected by this are C and C++, tools and dynamic
reports could be used to detect such vulnerability before shipping code to
production.

\section{Task 2}
\label{lab4-task2}

\subsection{CA-8 Penetration Testing}
\label{lab4-sub-1}

Penetration testing is an assessment run on systems to pinpoint weak points that
malicious actors can exploit and use. It can also be used to validate the
resilience within specified constraints and is a practice that is usually
conducted by an experienced team with a high level of knowledge, experience and
skills on various thematic such as network, operating system and application
security. An organisation usually use the results of a vulnerability analysis to
then engage in penetration testing activities that will then be conducted after
an agreement to the rules of engagement.

\subsection{PE-18 Location of System Components}
\label{lab4-sub-2}
The position of the system components needs to comply with order to minimise
damage from a defined physical and environmental hazards such as nature
disasters, terrorism, vandalism and more. Additionally, organisations need to
carefully appoint physical entry points to mitigate or minimise, unauthorised
access to the building where the system components are available and its
proximity, as the adversaries could use sniffers and microphones for espionage.

\subsection{PE-3 Physical access control}
\label{lab4-sub-3}
The position of the system components needs to comply to minimise damage from
defined physical and environmental hazards such as nature disasters, terrorism,
vandalism and more. Additionally, organisations need to carefully appoint
physical entry points to mitigate or minimise unauthorised access to the
building where the system components are available and its proximity, as the
adversaries could use sniffers and microphones for espionage.

\subsection{SC-5 Denial-of-service Protection}
\label{lab4-sub-4}
The Denial of Service protection controls aim to prevent and control the effects
of such attacks by adding a layer of security specifically crafted and setup by
an organisation. In the modern days, organisations usually prevent the
disruptions of availability on their services placing a powerful load balancer
in front of their services that scales very highly or pay a combination of
additional services such as AWS Shield coupled with CloudFront, Route 53 and WAF
for the most up to date protection to these kind of attacks.

\subsection{SI-10 Information Input Validation}
\label{lab4-sub-5}

This control checks if the system inputs have a valid syntax and semantics and
verifies that the data is interpreted as per business logic. This is to avoid
adversary attacks that could introduce and construct malicious commands that
would result in wrong interpretation and output.

\subsection{AT-2 Literacy Training And Awareness}
\label{lab4-sub-6}
Organisations will have to deliver basic and advanced training to their
employees, including tests to measure the understanding and the general
knowledge. The trainings are usually tailored based on the access level of each
individual or group of employees. There must be communications and small
training also based on recent changes such as important policies or changes
organisation security.

\section{Conclusion}
\label{lab4-conclusion}
This lab needed a lot of research and writing to complete but it was a nice
exploration. It helped me understand the relationship between vulnerabilities,
their impact and analyse each of them in detail with mitigation in mind.
